\section{Conclusion}
\label{sec:pureshift__conclusion}

This chapter revolves around the technique of pure shift NMR, which yields spectra where couplings are suppressed and only chemical shift information is shown.
Pure shift spectra provide significantly improved resolution, but with poorer sensitivity; the resulting spectra also suffer from artefacts due to detection of unwanted CTPs, especially in the case of strong coupling.
Various forms of the pure shift element were studied, with the aim of improving on the sensitivity and purity of the PSYCHE method\autocite{Foroozandeh2014ACIE}.
\cref{sec:pureshift__nsaltire} describes PSYCHE with either one or four saltire pulses, as opposed to the usual two;
in \cref{sec:pureshift__optimisation}, these saltire pulses are entirely replaced with an arbitrary shaped pulse obtained through optimisation;
\cref{sec:pureshift__timerev} looks at the time-reversal element, where different spectra are co-added to produce a pure shift spectrum;
and \cref{sec:pureshift__dpsyche} is devoted to the dPSYCHE method, where pulses and gradients are interleaved instead of being applied simultaneously.

Many of these did not yield any improvement over PSYCHE itself, and were thus not considered further.
In particular, the summation process of the time-reversal element introduced further spectral artefacts arising from incomplete cancellation of unwanted signals.
The most promising of these appears to be the dPSYCHE experiment: although I did not complete my investigation into this, the results obtained to date suggest that the experiment may be optimised to provide comparable decoupling quality to PSYCHE but with better sensitivity (see e.g.\ \cref{tbl:specopt_to_states}).
There is a clear path towards this end, starting with investigating the use of a different (state transfer-based) cost function; it is also worth increasing the flexibility of the optimisation, for example, by increasing the number of pulses or by optimising the gradient amplitudes as well, to see if better results can be attained.

Assuming that this can be accomplished, though, one may rightly still ask whether a slight 10--20\% improvement in signal-to-artefact ratio is \textit{useful}.
PSYCHE already allows the experimentalist to control this quantity by adjusting the flip angle of the saltire pulses, and thus the same outcome may in principle be obtained by lowering the flip angle (and increasing the measurement time to compensate for the drop in signal-to-noise).
Pure shift NMR is admittedly not a particularly widely used technique, and in my view, it is not for the lack of good pure shift elements.
It would be very much worthwhile exploring other potential areas in which it can be applied with greater regularity (or at all): an obvious candidate would be natural product assignment, where spectra are often highly complex, and incorrect assignments more abundant than one might like.\autocite{Nicolaou2005ACIE}

In the final section (\cref{sec:pureshift__epsidosy}), I also briefly discussed some preliminary efforts towards an ultrafast version of the PSYCHE-iDOSY experiment\autocite{Foroozandeh2016ACIE}.
Although the pulse sequence cannot be considered to be ready for use, the data collected so far show that the underlying concept is sound.
The use of ultrafast techniques to record a PSYCHE-iDOSY is mainly attractive due to the potential time savings, as a pseudo-3D experiment is reduced to a pseudo-2D experiment.
However, this comes with a significant reduction in sensitivity.
The use of pure shift already incurs a sensitivity penalty, and the addition of the ultrafast component further diminishes the available signal.
Thus, such a method would likely be applicable only to very concentrated samples.

While the work described in this chapter did not lead to published output, it did provide a motivating context for the idea of \textit{optimising NMR experiments}---particularly for pulse sequences which are expensive to simulate, such as PSYCHE.
In this chapter, the aim was purely to search for the best-performing pure shift sequence.
However, the dual considerations of sensitivity and purity are relevant to almost all NMR experiments; it is therefore natural to ask how experiments can be set up to maximise these criteria.
To answer this, I designed a software package called POISE: this is precisely the subject of the next chapter, which we now turn to.
