\subsection{Speeding up dPSYCHE simulations}
\label{subsec:pureshift__dpsyche_simulations}

To begin, I first explain how the exact simulation of dPSYCHE experiments can be greatly accelerated through efficient propagator calculations.
Although Spinach\autocite{Hogben2011JMR} is the leading simulation package for NMR, and covers an extremely impressive range of experiments, this generality also prevents it from providing optimal performance for any single experiment.
As it turns out, handwritten, specialised Matlab code can outperform Spinach by orders of magnitude.

The NMR simulations developed here simply use the density operator formalism in Hilbert space, as outlined in \cref{sec:theory__density_operators}.
The Zeeman basis is used, and non-unitary transformations such as relaxation are neglected.
Propagation under the Liouville--von Neumann equation (\cref{eq:lvn_interaction_integrated}) requires the calculation of matrix exponentials $\exp(-\mi H\tau)$.
For an $N \times N$ matrix, the matrix exponential requires $O(N^3)$ time to calculate (and for a system containing $p$ spins, we have $N = 2^p$); it is often this which is the bottleneck in NMR simulations.
Minimising the number of matrix exponentials, and/or their computational cost, is the key to achieving speedups, as will be shown in the following text.%
\footnote{Note that in my simulations, I simply used the builtin \texttt{expm} Matlab function, which implements the matrix exponential using a combination of the scaling-and-squaring method and Pad\'e approximation\autocite{Higham2005SIAMJMAA}.
This is in contrast to Spinach, which primarily uses a scaled-and-squared Taylor series (according to the \texttt{propagator.m} file, various other methods supposedly did not `live up to their marketing').
An in-depth discussion of matrix exponential methods is outside the scope of this thesis, but can be found in a classic paper by Moler and Van Loan\autocite{Moler2003SIAMR}.}

Generally, the accurate simulation of pulsed field gradients requires the sample to be divided up into $n$ discrete slices, each simulated with a different $\Hgrad$.\footnote{In simple cases this can be avoided by simply removing all terms with the wrong coherence orders as we know they will be dephased (\cref{eq:generalised_rephasing_2}), but this is too naive an approach for pure shift techniques.}
Thus, a very naive implementation of the dPSYCHE experiment would require $mn$ matrix exponentials, one per pulse per slice.
The overall structure of the code would resemble \cref{lst:dpsyche_slow}.%
\footnote{Strictly speaking, there is a slight inaccuracy in this code: the final gradient should have strength \texttt{-2G} and not \texttt{G}, but that is a minor detail which I leave out for clarity in the code.}
Note that the term \texttt{H\_free}, and the corresponding propagator \texttt{U\_free}, technically refer to the interaction-picture free Hamiltonian, $\HfreeI$.

\begin{mylisting}[htb]
    \centering
\begin{tcbminted}{matlab}
% loop over slices
for slce=1:n
    H_grad = I_z * G * z(slce);

    % loop over pulse points
    for pulse_point=1:m
        H_pulse = (c_x(pulse_point) * I_x) + (c_y(pulse_point) * I_y);

        % calculate propagators; m*n total matrix exponentials
        U_pulse = expm(-1j * (H_free + H_pulse) * t_pulse);
        U_grad = expm(-1j * (H_free + H_grad) * t_grad);

        rho = U_grad * U_pulse * rho * U_pulse' * U_grad';
    end
end
\end{tcbminted}
\caption[Naive dPSYCHE code]{Rough structure of a naive dPSYCHE implementation. Note that I use the variable name \texttt{slce} as \texttt{slice} is an existing builtin Matlab function.}
\label{lst:dpsyche_slow}
\end{mylisting}

It is not difficult to come up with a more sensible approach which cuts this down by a factor of $m$: since the pulses are not applied together with the gradients, the pulse propagators $U_{\symup{pulse}}$ can be pre-calculated outside of the loop.
Furthermore, all of the gradients within the PSE are the same, so $U_{\symup{grad}}$ can be moved out of the inner loop (\cref{lst:dpsyche_fast1}).

\begin{mylisting}[htb]
    \centering
\begin{tcbminted}{matlab}
% precalculate pulse propagators; m total matrix exponentials
for pulse_point=1:m
    H_pulse = (c_x(pulse_point) * I_x) + (c_y(pulse_point) * I_y);
    U_pulse(m) = expm(-1j * (H_free + H_pulse) * t_pulse);
end

% loop over slices
for slce=1:n
    % calculate gradient propagators; n total matrix exponentials
    H_grad = I_z * G * z(slce);
    U_grad = expm(-1j * (H_free + H_grad) * t_grad);
    
    % loop over pulse points
    for point=1:m
        rho = U_pulse(m) * rho * U_pulse(m)';
        rho = U_grad * rho * U_grad';
    end
end
\end{tcbminted}
\caption[Slightly faster dPSYCHE code]{Rough structure of a slightly faster implementation of dPSYCHE.}
\label{lst:dpsyche_fast1}
\end{mylisting}

Spinach, which is designed to be general, has no idea that these optimisations are possible, so is naturally rather slower.
However, even this is relatively inefficient.
It can be shown that the two components of the gradient propagator, $\HfreeI$ and $\Hgrad$, actually commute with one another (even in the strong coupling case).
Thus, we can write:
\begin{equation}
    \label{eq:split_propagator}
    \exp[-\mi(\HfreeI + \Hgrad)\tau] = \exp(-\mi\HfreeI\tau) \exp(-\mi\Hgrad\tau)
\end{equation}
(in general, for matrices $A$ and $B$, $\exp(A + B) = \exp(A)\exp(B)$ if and only if $[A, B] = 0$).
This on its own does not reduce the number of matrix exponentials required, but notice now that $\Hgrad$ is a sum of $I_{iz}$ terms and is therefore \textit{diagonal} in the Zeeman basis.
The exponential of a diagonal matrix is almost trivial to calculate, as we only need to exponentiate the diagonal \textit{elements:}
\begin{equation}
    \label{eq:expm_diagonal}
    \exp
    \begin{pmatrix}
        \lambda_1 & 0 & \ldots & 0 \\
        0 & \lambda_2 & \ldots & 0 \\
        \vdots & \vdots & \ddots & \vdots \\
        0 & 0 & \ldots & \lambda_n \\
    \end{pmatrix}
    = 
    \begin{pmatrix}
        \exp(\lambda_1) & 0 & \ldots & 0 \\
        0 & \exp(\lambda_2) & \ldots & 0 \\
        \vdots & \vdots & \ddots & \vdots \\
        0 & 0 & \ldots & \exp(\lambda_n) \\
    \end{pmatrix}.
\end{equation}
Instead of using the $O(N^3)$ \texttt{expm(M)} function, this can instead be done in $O(N)$ time using \texttt{diag(exp(diag(M)))} (the \texttt{diag} Matlab function converts a diagonal matrix to a vector of its diagonal entries, and vice versa).
So, we now require only $m + 1$ `true' matrix exponentials.
At this point, our matrix exponentials have almost been eliminated and the largest remaining bottleneck is almost certainly the matrix \textit{multiplications} required for the propagation.
We can cut these down by a factor of two simply by calculating the overall propagator
\begin{equation}
    \label{eq:overall_propagator}
    U_{\symup{total}} = U_n\cdots U_2U_1
\end{equation}
and then performing the propagation only at the very end:
\begin{equation}
    \label{eq:overall_propagation}
    \rho = U_{\symup{total}} \rho_0 \adj{U_{\symup{total}}}
\end{equation}
instead of performing every individual step $\rho \to U_1\rho_0\adj{U_1}$.
The final optimised code therefore resembles that in \cref{lst:dpsyche_fast2}.

\begin{mylisting}[htb]
    \centering
\begin{tcbminted}{matlab}
% precalculate propagator due to free evolution during gradient
% only 1 matrix exponential
U_free = expm(-1i * H_free * t_grad);

% precalculate pulse propagators; m total matrix exponentials
for pulse_point=1:m
    H_pulse = (c_x(pulse_point) * I_x) + (c_y(pulse_point) * I_y)
    U_pulse(m) = expm(-1j * (H_free + H_pulse) * t_pulse)
end

% loop over slices
for slce=1:n
    % initialise propagator for this slice
    U_slce = eye(2 ^ p);

    % calculate gradient propagators; no matrix exponentials required
    H_grad = I_z * G * z(slce);
    U_grad = U_free * diag(exp(diag(-1j * H_grad * t_grad)));

    % loop over pulse points
    for point=1:m
        U_slce = U_pulse(m) * U_slce;
        U_slce = U_grad * U_slce;
    end

    % perform propagation only at the end
    rho = U_slce * rho * U_slce';
end
\end{tcbminted}
\caption[Fast dPSYCHE code]{Rough structure of a fast dPSYCHE implementation.}
\label{lst:dpsyche_fast2}
\end{mylisting}

The performance of this handwritten code as compared to Spinach is summarised in \cref{tbl:dpsyche_simulations}.
In all cases, the spectra produced by the two methods were entirely equivalent.
It should be noted that the handwritten code does not even utilise CPU parallelisation, whereas Spinach does.
I investigated the possibility of parallelising the loop over slices (replacing the outer \texttt{for} in \cref{lst:dpsyche_fast2} with \texttt{parfor}): however, this in fact made the code \textit{slower}, presumably due to overhead.
This is a good thing: it means that \texttt{parfor} can be used in an external loop, e.g.\ for the parallel simulation of the dPSYCHE experiment on different spin systems.

\begin{table}[htb]
    % matlab_nmr_jy/research/compare_dpsyche.m
    \begin{tabular}{cccc}
        \toprule
        \textbf{Number of spins} & \textbf{Number of couplings} & \multicolumn{2}{c}{\textbf{Execution time (s)}} \\
        \cmidrule{3-4}
                                 & & Spinach & Handwritten \\
                                 \midrule
        \multirow{1}{*}{1}       & 0 & 3.33 & 0.35 \\
        \midrule
        \multirow{2}{*}{2}       & 0 & 4.59 & 0.32 \\
                                 & 1 & 6.22 & 0.32 \\
                                 \midrule
        \multirow{4}{*}{3}       & 0 & 9.90  & 0.43 \\
                                 & 1 & 12.95 & 0.45 \\
                                 & 2 & 31.86 & 0.47 \\
                                 & 3 & 35.01 & 0.48 \\
                                 \midrule
        \multirow{7}{*}{4}       & 0 & 30.59  & 1.02 \\
                                 & 1 & 38.63  & 0.99 \\
                                 & 2 & 44.61  & 1.04 \\
                                 & 3 & 365.04 & 1.52 \\
                                 & 4 & 446.03 & 1.57 \\
                                 & 5 & 521.72 & 1.57 \\
                                 & 6 & 588.31 & 1.69 \\
        \bottomrule
    \end{tabular}
    \caption[Comparison of wall-clock times for dPSYCHE simulations]{
        Comparison of wall-clock times for dPSYCHE simulations.
        The dPSYCHE sequence used contained 15 pulses, each applied with a flip angle of \ang{15} and a phase of \ang{0}.
        Spin systems were generated pseudo-randomly.
        All timings were measured on a 2019 MacBook Pro with a \qty{2.6}{\GHz} 6-core Intel i7 CPU (although CPU parallelisation was not used).
    }
    \label{tbl:dpsyche_simulations}
\end{table}
