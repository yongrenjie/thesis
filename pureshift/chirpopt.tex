\subsection{Waveform parameterisation and optimisation}
\label{subsec:pureshift__chirpopt}

Given that a working optimisation setup, including cost functions, had been developed, it was a logical step to then test it out on a more challenging problem: namely, how the waveform used in the PSYCHE PSE could be modified.
This goes beyond simply modifying the number of saltires, as was done in \cref{sec:pureshift__nsaltire}.
There is no real reason why the pulse \textit{must} be an integer number of saltires: in principle it can have \textit{any} shape, although being symmetric about the centre of the pulse would likely still be beneficial in terms of preserving the mechanism of spatiotemporal averaging.

A naive attempt at optimising the pulse would simply involve modifying every pulse point in the double-saltire waveform used in the PSYCHE element.
As described in \cref{sec:theory__rotating_frame}, each pulse point consists of a pair of $x$- and $y$-amplitudes $(c_x, c_y)$; therefore, for a pulse with $n$ points, we would have a parameter vector $\symbf{x} \in \mathbb{R}^{2n}$.
Unfortunately, for PSYCHE, $n$ is on the order of $10000$, and an optimisation with $20000$ points is totally unfeasible.%
\footnote{Although problems of this size have been tackled using optimal control theory\autocite{Khaneja2005JMR,deFouquieres2011JMR,Glaser2015EPJD,Goodwin2016JCP}, it is not really feasible to use it in cases where the pulse is applied \textit{together} with a gradient, as is the case in PSYCHE. On top of that, the coupling networks and spin systems of interest are rather more complicated than in typical applications of optimal control.}

As a result of this, we must consider other ways of parameterising the waveform.
Several approaches to this issue have surfaced in the literature, such as the use of Fourier series\autocite{Geen1991JMR,Kupce1995JMRSB}, Gaussian cascades\autocite{Emsley1990CPL}, or spline interpolation between a subset of pulse points\autocite{Ewing1990CP}.
In this instance, we can use the knowledge that the PSYCHE PSE is composed of saltire pulses to our advantage.
Each saltire pulse is a linear combination of two chirps, defined by:
\begin{align}
    \phi(t) &= \phi_0 + \pi\taup(\Delta F)\left(\frac{t}{\taup} - \frac{1}{2}\right)^2 \label{eq:chirp_pulse_phase} \\
    c_x(t) &= f_\text{smooth}(t) A \cos[\phi(t)] \label{eq:chirp_pulse_cx} \\
    c_y(t) &= f_\text{smooth}(t) A \sin[\phi(t)] \label{eq:chirp_pulse_cy}
\end{align}
for $t \in [0, \taup]$. Here, $\taup$ is the duration of the chirp, $A$ is the amplitude of the chirp (which is time-independent), $\phi_0$ the phase of the chirp, and $\Delta F$ the bandwidth.
\Cref{eq:chirp_pulse_cx,eq:chirp_pulse_cy} are identical to \cref{eq:pulse_cartesian}, except for the addition of a \textit{smoothing function} $f_\text{sm}(t)$, which prevents large jumps in RF amplitude at the beginning and end of the pulse.
$f_\text{sm}$ depends on a smoothing parameter $s_\text{sm}$, which is typically $0.1$--$0.2$:
\begin{equation}
    \label{eq:sming_function}
    f_\text{sm}(t) = \begin{cases}
        \displaystyle \sin\left(\frac{\pi t'}{2 s_\text{sm}}\right) & 0 \leq t' < s_\text{sm}; \\
        \displaystyle 1 & s_\text{sm} \leq t' < 1 - s_\text{sm}; \\
        \displaystyle \sin\left[\frac{\pi (1 - t')}{2 s_\text{sm}}\right] & s_\text{sm} \leq t' \leq 1, \\
    \end{cases}
\end{equation}
where $t' = t/\taup$.

\todo{(could do with a figure here)}

Given these expressions, and ignoring the smoothing function (which is only really described here for completeness) we see that there are four parameters of the chirp which can be modified: $A$, $\taup$, $\phi_0$, and $\Delta F$.
The two chirps which form one saltire pulse simultaneously sweep in opposite directions, which mean that $\Delta F$ for one chirp is the negative of the other; however, their parameters are otherwise equal.

We may, however, also envision a case where the pulse is constructed from two chirps which are applied at a different point in time, and also cover different bandwidths.
This adds two more parameters to each chirp, namely $t_0$ (the starting time of the pulse) and $f_0$ (the centre of the pulse bandwidth).
Each chirp therefore sweeps from the frequency $f_0 - (\Delta F)/2$ at a time $t_0$, to the frequency $f_0 + (\Delta F)/2$ at a time $t_0 + \taup$.
In total, this gives us six parameters per chirp which may be optimised: for a sum of two chirps, there are therefore 12 parameters in total.
Since a sum of two chirps is not necessarily symmetric (with respect to reflection in time), I opted to reflect the waveform about its end, thus doubling the length of the pulse.

The initial point chosen was that corresponding to a \textit{single} saltire after reflection:
\begin{itemize}
    \item Chirp 1: $\taup = \SI{15}{\ms}$; $\Delta F = \SI{-5}{\kHz}$; $\phi_0 = 0$; $A = \SI{36}{\Hz}$; $t_0 = 0$; $f_0 = \SI{5}{\kHz}$;
    \item Chirp 2: $\taup = \SI{15}{\ms}$; $\Delta F = \SI{5}{\kHz}$; $\phi_0 = 0$; $A = \SI{36}{\Hz}$; $t_0 = 0$; $f_0 = \SI{-5}{\kHz}$;
\end{itemize}
Collectively, these two pulses sum up to become the \textit{first half} of a single saltire with bandwidth $\SI{10}{\kHz}$.
After reflection, the total duration of the saltire is $\SI{30}{\ms}$, and the amplitude is $\SI{72}{\Hz}$, which corresponds to a flip angle of approximately \ang{32}.

Using this setup, several optimisations of the 12 parameters above were conducted.
It was quickly noticed that, although $f_\text{diff}$ was a better cost function than $f_\text{phase}$, this led to spurious optima being located.
\todo{(FIGURE)}
The reason for this is almost certainly because the PSE distorts the relative intensity of signals: in particular, singlets have lower relative intensities than in the parent pulse--acquire spectrum.
Of course, singlets are completely unimportant when devising a pure shift experiment.
Furthermore, since singlets are typically more intense than the rest of the spectrum, they also contribute disproportionately to the cost function.

A simple and effective way to avoid this is to evaluate the cost function only on a specific part of the spectrum.
In this case, I chose to use the $\ch{H}^{\alpha}$ region between 4.72 and \SI{5.94}{ppm}

\todo{(figure -- spurious optima are gone, we have a \texttt{chirpopt\_spurious.py} file)}

\todo{(even with this, often converged to saltire... on the occasion something better than a saltire was found, typically the result is very close to that of a saltire --- show TSE-PSYCHE spectrum obtained, 190831 AV600 cyclosporin)}

While not quite as intractable as 20000 parameters, optimising a 12-parameter function clearly still proves to be a challenge.
Although the cost functions described here do work, they are generally quite `flat', in that they do not discriminate very sharply between `good' and `bad' spectra.
Combined with the fact that the cost function is noisy, this makes experimental optimisation of the waveform an uphill task.
With that said, much of the knowledge (and code) in this section was later used in the development of POISE.
In particular, PSYCHE optimisations also ended up being part of POISE (\cref{subsec:poise__psyche}).
