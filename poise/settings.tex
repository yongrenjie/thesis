\subsection{Optimisation settings}
\label{subsec:poise__settings}

Once the user has defined a routine, it can then be run from the TopSpin command line using the command \texttt{poise ROUTINE\_NAME}.
However, the routine itself merely controls what parameters are being optimised: it does not specify what experiment is to be run (i.e.\ the pulse programme), nor any of the other parameters in the experiment.
These must be set by the user, and can most conveniently be stored in a TopSpin parameter set which can simply be loaded before starting the optimisation.
This flexibility means that the same \textit{type} of optimisation may be applied to different pulse sequences without having to create individual routines for each: for example, an experiment to optimise the NOE mixing time (as described in \cref{subsec:poise__noe}) can be run with different versions of the NOESY sequence depending on what is most appropriate.
Likewise, parameters such as the number of scans can be adjusted in order to run optimisations on samples with different concentrations.

Once the experiment parameters have been set up, there are a few more options which control how the optimisation is carried out:

\begin{itemize}
    \item the \texttt{--maxfev} option allows the user to control the maximum number of FEs, or in other words, the maximum number of experiments run.
        If the optimisation has not converged after acquiring this many spectra, the best result so far is simply returned.
        This effectively allows the time spent on optimisation to be capped.

    \item the \texttt{--quiet} option silences all output from the optimiser (the best parameters found are stored in the dataset itself after the optimisation ends, and can therefore be retrieved).
        This is useful when a POISE optimisation is to be run under automation.
        
    \item the \texttt{--separate} option allows each FE to be run in a new experiment number, so that the optimisation trajectory can be analysed after its conclusion.

    \item perhaps most importantly, the \texttt{--algorithm} option allows the user to choose one of three optimisation algorithms.
        I now describe these algorithms in greater detail.
\end{itemize}
