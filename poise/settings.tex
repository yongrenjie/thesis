\subsection{Optimisation options}
\label{subsec:poise__options}

Finally, POISE provides a few command-line options which control how the optimisation is carried out:

\begin{itemize}
    \item the \texttt{-\phantom{}-maxfev} option allows the user to control the maximum number of FEs, or in other words, the maximum number of experiments run.
        If the optimisation has not converged after acquiring this many spectra, the best result so far is simply returned.
        This indirectly allows a cap to be placed on the time spent on optimisation.

    \item the \texttt{-\phantom{}-quiet} option silences all output from POISE.
        This is useful when, for example, a optimisation is to be run under automation.
        (The optima found are stored in the dataset itself after the optimisation ends, and can therefore always still be retrieved.)
        
    \item the \texttt{-\phantom{}-separate} option allows each FE to be run in a new experiment number, so that the optimisation trajectory can be analysed after its conclusion.

    \item perhaps most importantly, the \texttt{-\phantom{}-algorithm} option allows the user to choose one of three optimisation algorithms, which I now describe in more detail.
\end{itemize}
