\subsubsection{Multidirectional search}

In the preceding discussion, we noted that the simplex $Y$ was nonsingular if the $n$ vectors $y_k - y_1$ were linearly independent.
Equivalently, the matrix $M$ formed by concatenating these vectors
\begin{equation}
    \label{eq:simplex_matrix}
    M = \begin{pmatrix}
        y_2 - y_1, y_3 - y_1, \cdots, y_{n+1} - y_1 \\
    \end{pmatrix}
\end{equation}
must be nonsingular, i.e.\ have a nonzero determinant.
We can quantify how `close' the simplex $Y$ is to being singular, using the $l^2$ condition number of the matrix $M$, which in this context is usually referred to as the \textit{simplex condition}:
\begin{equation}
    \label{eq:simplex_condition}
    \kappa(Y) = \lVert M \rVert \lVert M^{-1} \rVert,
\end{equation}
where $\lVert M \rVert$ is the matrix norm induced by the Euclidean norm,
\begin{equation}
    \label{eq:matrix_norm}
    \lVert M \rVert = \max_{x \neq 0} \frac{\lVert Mx \rVert}{\lVert x \rVert}.
\end{equation}
A singular simplex $Y$ of course does not have a well-defined condition, since $M^{-1}$ does not exist.
However, the larger the condition of a simplex is, the closer it is to being singular.
Very loosely speaking, a long and thin simplex has a large condition number, and would be singular if its width were to go to zero.

The simplex updates made in the process of the NM algorithm mean that the simplex condition changes throughout the course of the optimisation.
This is good for achieving rapid decreases in the cost function, since the simplex shape adapts to the cost function being optimised.
However, if the simplex condition gets too large, it is possible that the optimisation will stall at a nonstationary point, since the search directions of the simplex are severely limited.
The MDS algorithm was proposed partially for the purpose of avoiding this ill-conditioning.\footnote{The main reason was in fact to better exploit computer parallelism, but it was also noticed that the MDS method proved to be generally more robust than NM.}

The MDS method is also simplex-based, and uses similar reflection/expansion/contraction steps to the NM algorithm.
However, instead of (e.g.) reflecting a single worst point $y_{n+1}$ about the other points, it reflects \textit{all} of the $n$ worst points $\{y_2, y_3, \cdots, y_{n+1}\}$ about the best point $y_1$.
This means that the shape of the simplex $Y$, and thus its condition $\kappa(Y)$, is always preserved: this provides it with much better convergence properties, as was shown by Torczon et al.\autocite{Torczon1989,Torczon1991SIAMJO}%
\footnote{Specifically, it can be concluded that at least one of the search directions was bounded away from being orthogonal to the gradient; or in simpler (and less precise) terms, at least one of the search directions is close enough to a direction in which the cost function $f\/$ decreases.}
The increased reliability of the MDS algorithm over the NM algorithm was demonstrated on a variety of example optimisation problems: even in the very simple case where the cost function was simply the norm of a vector,
\begin{equation}
    \label{eq:norm_cf}
    f(y) = \lVert y \rVert,
\end{equation}
it was shown that the NM algorithm stalled when the dimension of the problem, $n$, was sufficiently large.
The value of $n$ needed to precipitate this failure depended on the problem being solved, and generally ranged from 8 to 40.
On the other hand, the MDS method proved to be robust under the same conditions, eventually converging to the optimum---although in the cases where NM \textit{did} work, the MDS method generally required more FEs.

It was this improved robustness of the MDS algorithm which prompted Goodwin et al.\ to use it in their (experimental) optimisation of ESR pulse shapes\autocite{Goodwin2018JMR}, and me to later include it in POISE.
In the ESR work, the number of pulse points being optimised was 11 or 21, which fell into the regime where the MDS method would likely have better convergence properties than NM.
However, in NMR, each experiment takes a substantially longer time than in ESR, and even optimisations with $n > 2$ become rather time-consuming due to the number of FEs required (the largest $n$ explored in the present work is 4).
Thus, it is unlikely that the NMR optimisations explored in POISE will lie in the regime where the NM algorithm is prone to failure.
As will be shown later, I generally found that the NM and MDS methods were equally reliable in POISE optimisations, with NM generally being faster.
