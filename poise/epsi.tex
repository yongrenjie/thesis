\subsection{Ultrafast NMR}
\label{subsec:poise__epsi}

The final example of a single-parameter POISE optimisation is also the most complicated: it pertains to the EPSI acquisition technique, which was previously introduced in \cref{sec:pureshift__epsidosy}.
EPSI acquisition allows signals from different slices of the sample to be simultaneously detected in a single FID, and (in liquid-state NMR) has most famously been used in the context of \textit{ultrafast} 2D experiments\autocite{Frydman2002PNASUSA,Pelupessy2003JACS,Frydman2003JACS,Tal2010PNMRS,Giraudeau2014ARAC,Gouilleux2018ARNMRS}, where the $t_1$ evolution is spatially encoded and read out using the EPSI technique.

During an EPSI acquisition period, alternating gradients of equal magnitude but opposite sign are applied, as shown in \todo{FIG -- zgepsi and tocsyepsi}.
Each data point collected therefore depends not only on $t_2$ (the acquisition time domain) but also on $k$, a wavenumber which measures the extent of gradient dephasing caused by the acquisition gradients:
\begin{equation}
    \label{eq:epsi_k_space}
    k = \int_0^{t_2} \gamma G(t) \,\mathrm{d}t.
\end{equation}
This value of $k$ increases during positive gradients and decreases during negative gradients, leading to a zigzag pattern in $k$-space, all while $t_2$ is still increasing (\todo{FIG -- data points, correct and incorrect})
The \textit{prephasing gradient} immediately before acquisition has the same duration as the EPSI gradients, but has half the amplitude, meaning that $k$ begins from a negative value, specifically, $k_\text{max} / 2$, where $k_\text{max}$ is the change in $k$ caused by one complete positive EPSI gradient.
As shown in previous reviews\autocite{Frydman2003JACS}, the spatial profile $f(z)$ can be obtained by Fourier transformation of the $k$ domain.
Alternatively, since $t_1$ is proportional to $z$ and the indirect-dimension frequency $F_1$ is itself obtained through Fourier transformation of the $t_1$-domain, the $F_1$ and $k$ domains are in fact directly proportional: no Fourier transform is required along this axis to obtain the indirect-dimension frequencies.

This rapid alternating of gradients is very demanding on spectrometers, and it is often the case that the positive and negative EPSI gradients---although \textit{nominally} specified with the same amplitude---are not perfectly balanced.
This causes a `drift' in the $k$-domain over time, as illustrated in \todo{FIG}.
In this section, POISE is used to perform an \textit{instrument-specific} optimisation in order to correct for this effect.

\todo{change axis order of $k,t_2$ --- Frydman always uses a weird depiction for whatever reason...}


\subsubsection{Optimisation setup}

To measure the drift in $k$-space, it is easiest to use a pulse sequence for which there is no spatial encoding; the pulse--EPSI experiment (\todo{FIG}) is perfectly suited for this.
In the pulse programme, I multiplied the amplitudes of the negative gradients by a factor $\alpha$, represented as \texttt{CNST16} in TopSpin:
the objective of POISE is therefore to determine the optimum value for this which minimises the drift.
Calculating this drift requires fairly substantial processing, which is most easily done inside the cost function using \texttt{numpy} functions (rather than, for example, a TopSpin AU script).
This consists of the following steps:
\begin{enumerate}
    \item reading in the 1D FID and reshaping it into a 2D $(k, t_2)$ matrix;
    \item discarding data obtained during negative gradients;
    \item determining, for each point in $t_2$, the value of $k$ for which the maximum signal is found;
    \item performing linear regression on these points and obtaining the slope of $k$ against $t_2$, which directly represents the drift in $k$-space.
\end{enumerate}
