\section{Conclusion}
\label{sec:poise__conclusion}

In this chapter, I described the development of POISE, a Python programme for optimising any set of (numeric) acquisition parameters in Bruker's TopSpin software.
This allows experiments to be tailored to the sample and instrument being used, and can be run in a completely automated fashion once the requisite information has been defined in a routine.

The examples of POISE optimisations shown in this chapter (\cref{sec:poise__applications}) include a variety of solution-state 1D and 2D NMR experiments.
However, because POISE was intentionally designed to be \textit{general}, it is easy to adapt it to essentially any NMR experiment, as long as a suitable cost function can be constructed.
Thus, it is not difficult to envision POISE being used in other fields such as biomolecular and solid-state NMR, and extensions in this direction are worth pursuing: the idea of applying POISE to the cross-polarisation technique\autocite{Pines1972JCP} was briefly floated, but never actually explored.
More broadly speaking, real-time experimental optimisation can be extended to other forms of spectroscopy: the successful development of ESR-POISE (\cref{sec:poise__esrpoise}) bears witness to this, although this is rather more involved as it entails writing instrument- and manufacturer-specific code.

Returning to the confines of solution-state NMR, though, it is worth considering the types of situations in which POISE is likely to find most use.
In this chapter, I have asserted that POISE is \textit{fast}, being able to run optimisations in a matter of seconds to minutes.
However, perhaps with the exception of the pulse width calibration in \cref{subsec:poise__pulsecal}, I did not perform any comparisons (either in terms of time, or the quality of the optimum found) against a `skilled user' such as an experienced spectroscopist.
Without such data, I cannot justly claim that POISE should be used to replace an expert.
Nonetheless, there are many situations in which experts are not available: most notably, when samples are run in a high-throughput `automation' manner.
In my view, it is these situations in which an optimisation programme can best be used: the POISE routine need only be set up once, and can thereafter be incorporated into data acquisition procedures.
A simple example would be to determine $T_1$ values for a compound (which, as shown in \cref{subsec:poise__invrec}, takes 2--3 minutes), so that recovery delays in 2D experiments can be appropriately adjusted.
Other caveats regarding the usage of POISE have already been covered in \cref{sec:poise__notpoise}.
Even bearing these in mind, though, the results in this chapter form a strong argument for the inclusion of POISE in a spectroscopist's toolbox.
