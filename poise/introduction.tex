\section{Introduction}
\label{sec:poise__introduction}

In the previous chapter, I covered various approaches to improving pure shift NMR through the use of optimisation.
Although the optimisation code written there was highly specialised and only designed to work on pure shift applications, it was envisioned that this optimisation approach could be applied to essentially \textit{any} NMR experiment where parameter optimisation was required.
In principle, this description is appropriate for \textit{every} experiment: even the simplest pulse--acquire experiment can be optimised through the use of Ernst angle excitation.
More complex examples, such as 2D experiments, typically have parameters which should be chosen to optimally match coupling constants (INEPT delay) or relaxation rates (NOE mixing time).

In practice, the need for accurate parameters is often `solved' through the use of compromise values, which typically fall in the middle of an expected range for typical molecules.
For example, these values may be stored as part of a parameter set designed to be reused.
Alternatively, parameter values may be optimised `by hand'.
However, compared to these, the use of experimental optimisation has several benefits.
It is:
\begin{enumerate}
    \item \textit{sample-specific}, and as long as the default values are within the optimisation bounds, the optimisation will yield performance which is no worse than the defaults;
    \item more \textit{robust} towards unusual molecular structures, which have physical or chemical properties which fall outside of an expected range;
    \item \textit{instrument-specific}, so can compensate for spectrometer imperfections.
    \item \textit{automated}, so does not require an expert to adjust parameter values manually, or even any user intervention for that matter;
    \item \textit{objective}, in that the quality of a spectrum can (in principle) be mathematically measured through a cost function; and
    \item \textit{fast}, in that it uses an algorithm which is designed to achieve rapid decreases in the objective function: many `manual' optimisations involve either trial-and-error or an exhaustive grid search (i.e.\ increasing a parameter value one step at a time), neither of which are efficient.
\end{enumerate}

Despite these advantages, experimental optimisation of NMR parameters has seen only limited use.
In fact, although there are several examples of such optimisations in laser\autocite{Bardeen1997CPL}, nuclear quadrupole resonance\autocite{Schiano1999JMR,Schiano2000ZNA,Monea2020JMR}, and ESR\autocite{Goodwin2018JMR} spectroscopies, 
the only direct parallel in NMR which I have found is that of the eDUMBO pulses for heteronuclear\autocite{DePaepe2003CPL,Elena2004CPL} and homonuclear dipolar\autocite{Salager2010CPL} decoupling in solid-state MAS experiments.
In this work, the Emsley group used `direct spectral optimisation' (equivalent to what I call `experimental optimisation') to determine the best coefficients for a Fourier series pulse.
The performance of these pulses was measured by a cost function which (primarily) took into account the intensity of the detected peaks: a larger intensity corresponds to better decoupling performance.
Interestingly, the aim of using an experimental optimisation here was not to obtain sample-specific pulses (point (1)), but rather to account for the `spectrometer response', i.e.\ instrumental non-idealities (point (3)).
It was assumed that the compound used for the optimisation was a suitably representative choice, so that the optimisation result could simply be applied to other samples with no change.

The likely reason for the low popularity of experimental optimisations is \textit{time}.
In most cases, it is probably easier to run NMR optimisations in a theoretical manner, which can be much faster compared to the acquisition of a spectrum (depending on what simulations are involved), and also circumvents the effect of noise.
Examples of such optimisations include the design of shaped pulses, either through optimal control theory\autocite{Skinner2003JMR,Khaneja2005JMR,Kobzar2008JMR,Kobzar2012JMR,Schilling2014ACIE,Glaser2015EPJD} or by simple parameterisation\autocite{Geen1989JMR,Emsley1990CPL,Geen1991JMR,Nuzillard1994JMRSA,Kupce1995JMRSA,Kupce1995JMRSB}: these were briefly discussed in \cref{subsec:pureshift__chirpopt}.
(In fact, even the aforementioned eDUMBO pulses were not \textit{originally} designed as an experimental optimisation: they are actually an enhancement of the DUMBO decoupling schemes, which were optimised using numerical simulations\autocite{Sakellariou2000CPL}.)
It is also possible to design entire pulse sequences using \textit{in silico} optimisations\autocite{Shaka1985JMR,Freeman1987JMR,Bechmann2013JMR,Ehni2014JMR,Lapin2020JMR}: this is essentially what I did with the dPSYCHE experiment (\cref{sec:pureshift__dpsyche}).
However, doing this in an experimental fashion would almost certainly be prohibitively slow.

In this chapter, I aim to provide a convincing argument that experimental optimisation is not necessarily slow.
In particular, I will show that it is often possible to devise optimisation routines which yield improved results in a matter of minutes.
All the optimisations here are performed using a software package written by me, called POISE (Parameter Optimisation by Iterative Spectral Evaluation).
POISE is open-source (\url{https://github.com/foroozandehgroup/nmrpoise}) and can be installed in a single step through \texttt{pip install nmrpoise}.
Furthermore, it comes with extensive user documentation, both in the form of a text guide (\url{https://foroozandehgroup.github.io/nmrpoise}) as well as video (\url{https://www.youtube.com/watch?v=QTCeSCRZs4I}).

In contrast to previous work, which typically feature optimisations targeted at one specific application, I have endeavoured to make POISE as customisable and as broad as possible.
This generality is what allows a single software package, POISE, to perform all the optimisations described in this chapter; it also means that other users can devise specific cost functions and optimisation procedures for their own use.
Thus, \textit{POISE is more than just the applications shown later in this chapter}: it is really a platform which makes it possible to carry out arbitrary optimisations on an NMR spectrometer.
