\section{What POISE is not}
\label{sec:poise__notpoise}

Before moving on to cover applications of POISE, I want to make a note about several limitations of the approach chosen. 

Firstly, \textit{POISE is not specialised}.
While generality is a strength in that POISE can be applied to a diverse range of NMR experiments, it can also be a weakness.
POISE \textit{always} follows the framework in \cref{fig:poise_flowchart}: in particular, it simply seeks to find the optimum $\symbf{x}^*$, defined by
\begin{equation}
    \label{eq:poise_argmin}
    \argmin_{\symbf{x}} f(\symbf{x}).
\end{equation}
This rigidity in the underlying logic means that it is very conceivable that in specific instances, specialised optimisation routines which use customised strategies for data acquisition and analysis \textit{can} outperform POISE in terms of speed and/or accuracy.
For example, we see this in \cref{subsec:poise__pulsecal}: the TopSpin \texttt{pulsecal} routine for pulse width calibration can be much faster than POISE, because it only needs to perform one experiment to obtain an answer.

A related point is that on each FE, the only bits of information retained are the parameters $\symbf{x}$ and the value of the cost function $f(\symbf{x})$.
The spectral data itself is not stored anywhere:\footnote{In principle, it \textit{could} be. There is nothing stopping me from implementing something to store previous spectra; it was just not the original motivation behind POISE.} thus, it is not possible to perform (for example) an `optimisation' which collects scans until a certain SNR is reached, or one which collects $t_1$ increments of a 2D spectrum and performs non-uniform sampling (NUS) processing until the signal to artefact ratio is sufficiently high.
In particular, I want to distinguish POISE from other types of `optimisations' reported in the literature, which typically \textit{accumulate} data points until a given confidence level is reached (e.g.\ through a model-fitting procedure).
Such procedures have been performed before in the contexts of (for example) relaxation measurements\autocite{Song2018JMR,Tang2019SR} and undersampling in multidimensional NMR\autocite{Eghbalnia2005JACS,Hansen2016ACIE,BrukerSmartDriveNMR}.

Secondly, \textit{POISE is not a global optimiser}.
The optimisation algorithms provided within POISE are not designed to search for global minima (except for Py-BOBYQA, but as described in \cref{subsec:poise__algorithms}, I disabled the multiple restarts option responsible for this).
In challenging optimisation cases where multiple local minima exist, it is not generally possible to predict which local minimum the algorithm will converge to.
What \textit{can} be guaranteed is that if the initial point is not already an optimum, then the optimisation will always provide a decrease in the cost function: in other words, it will always lead to an improvement in the spectrum (insofar as the cost function accurately represents the quality of the spectrum).

Finally, \textit{POISE is not a panacea}.
It should be noted that there is always an inherent tradeoff against the time required for the optimisation itself.
For example, it makes little sense to spend several minutes optimising the sensitivity of a pulse--acquire experiment: the time could simply be used to improve the SNR by collecting more scans.
There is also the critical---though undeniably subjective---question of whether the optimisation is \textit{worth it}: even if better results can be obtained in relatively short times, does this provide a substantial benefit over a `compromise' value in a default parameter set?%
\footnote{Of course, even though it is nowadays fashionable for authors to imply that their publications possess \textit{great impact}, a similar argument can be applied to \textit{many} scientific discoveries. To use an example from the next chapter, is it really necessary to acquire NOAH spectra when one can just acquire the standalone 2D experiments? I have seen arguments on both sides---some people simply do not need the speedups provided and do not want to spend the time to set up or troubleshoot new experiments.}
I do not profess to have a definitive answer to this, and I leave the reader to form their own conclusions in the specific contexts where they may consider using POISE.
In any case, for practical use, it is imperative to make sure that the optimisation is either fast, or solves a problem which cannot simply be tackled through signal averaging in the same amount of time.
It is my hope that this is (broadly) true of the examples shown.
