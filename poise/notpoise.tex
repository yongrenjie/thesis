\section{What POISE is not}
\label{sec:poise__notpoise}

Before moving on to cover applications of POISE, I want to make a note about several limitations of the approach chosen. 

Firstly, \textit{POISE is not specialised}.
Generality is a strength in that POISE can be applied to a diverse range of NMR experiments.
However, it can also be a weakness, because the capabilities of POISE must be restricted such that it can be adapted to different scenarios.
POISE \textit{always} follows the framework in \cref{fig:poise_flowchart}: in particular, it simply seeks to find the optimum $\symbf{x}^*$, defined by
\begin{equation}
    \label{eq:poise_argmin}
    \argmin_{\symbf{x}} f(\symbf{x}).
\end{equation}
Although this conforms to the mathematical notion of an optimisation, there are many other ways of searching for ideal NMR parameters, such as fitting data to a model or directly reading parameters off a spectrum.

This rigidity in the underlying logic means that it is very conceivable that in specific instances, specialised optimisation routines which use customised strategies for data acquisition and analysis \textit{can} outperform POISE in terms of speed and/or accuracy.
For example, we see this in \cref{subsec:poise__pulsecal}: the TopSpin \texttt{pulsecal} routine for pulse width calibration can be much faster than POISE, because it only needs to perform one experiment to obtain an answer.
I particularly want to distinguish POISE from other types of `optimisations' reported in the literature, which typically \textit{accumulate} data points until a given threshold is reached (in terms of, say, SNR or confidence in a parameter).
Examples of such procedures can be found in the contexts of relaxation measurements\autocite{Song2018JMR,Tang2019SR} and undersampling in multidimensional NMR\autocite{Eghbalnia2005JACS,Hansen2016ACIE,BrukerSmartDriveNMR}.

Secondly, \textit{POISE is not a global optimiser}.
The optimisation algorithms provided within POISE are not designed to go beyond the first local minimum found (Py-BOBYQA technically can, but as described in \cref{subsec:poise__algorithms}, I disabled the multiple restarts option responsible for this).
In challenging optimisation cases where multiple local minima exist, it is not generally possible to predict which local minimum the algorithm will converge to.
What \textit{can} be guaranteed is that if the initial point is not already an optimum, then the optimisation will always provide a decrease in the cost function: in other words, it will always lead to an improvement in the spectrum (insofar as the cost function accurately represents the quality of the spectrum).

Finally, \textit{POISE is not universally applicable}.
It should be noted that there is always an inherent tradeoff against the time required for the optimisation itself.
For example, it makes little sense to spend several minutes optimising the sensitivity of a pulse--acquire experiment: the time could simply be used to improve the SNR by collecting more scans.
There is also the critical---though undeniably subjective---question of whether the optimisation is \textit{necessary}: even if an optimisation can be run in a relatively short time, are the results really significantly better than a `compromise' value in a default parameter set?%
\footnote{Of course, a similar argument can be applied to \textit{many} scientific advances. To use an example from the next chapter, is it really necessary to acquire NOAH spectra when one can just acquire the standalone 2D experiments? I have seen arguments on both sides---some people simply do not need the speedups provided and do not want to spend the time to set up or troubleshoot new experiments.}
I do not profess to have a definitive answer to this, and I leave the reader to form their own conclusions in the specific contexts where they may consider using POISE.
In any case, a `meaningful' optimisation is likely to either be fast, and/or solve a problem which cannot simply be tackled through signal averaging in the same amount of time.
It is my hope that this is true of (most of) the examples shown.
