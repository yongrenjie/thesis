\section{Quantum mechanics}
\label{sec:theory__quantum_mechanics}

The most fundamental equation in (non-relativistic) quantum mechanics, which governs the time evolution of a quantum state $\ket{\Psi(t)}$ under a Hamiltonian $H$, is the time-dependent Schr\"{o}dinger equation:
\begin{equation}
    \label{eq:tdse}
    \frac{\partial\!\ket{\Psi(t)}}{\partial t} = -\frac{\mi}{\hbar}H\ket{\Psi(t)} 
\end{equation}
For a Hamiltonian which is constant over a period of time $t_1 \leq t \leq t_2$ (i.e.\ is \textit{time-independent}), this can be integrated to yield an explicit solution:
\begin{equation}
    \label{eq:time_evolution}
    \ket{\Psi(t_2)} = \exp\left[-\frac{\mi H (t_2 - t_1)}{\hbar}\right]\ket{\Psi(t_1)}
\end{equation}
In NMR, it is conventional to use units of angular frequencies instead of energies, for example by replacing $H/\hbar \to H$; this will henceforth be assumed.
The term $\exp[-\mi H(t_2 - t_1)]$ is called the \textit{propagator} of the system and denoted $U(t_2, t_1)$; this is often further simplified to $U(\tau)$ where $\tau = t_2 - t_1$ is the duration of the evolution.
For a Hamiltonian which varies with time but is piecewise constant, in that it can be broken up into several finite periods within which $H\/$ is time-independent, the time evolution of the state is simply given by successive application of propagators:
\begin{equation}
    \label{eq:time_evolution_piecewise}
    \ket{\Psi(t_n)} = U(t_n,t_{n-1})\cdots U(t_2,t_1)U(t_1,t_0) \ket{\Psi(t_0)}
\end{equation}
where $t_n > t_{n-1} > \cdots > t_0$.
The case where $H\/$ continuously varies with time is more complicated, but we will not need to consider it in this thesis.

In NMR spectroscopy, we manipulate the \textit{spin angular momentum} of atomic nuclei in order to obtain information about chemical structure and dynamics.
The present work is restricted to nuclei with spin quantum number $I = 1/2$.
These are two-level systems, where the eigenstates of $I_z$ (denoted as $\ket{\alpha}$ and $\ket{\beta}$ for $m_I = +1/2$ and $-1/2$ respectively) are used as a standard basis, called the \textit{Zeeman basis}.
Primarily for mathematical convenience, the $z$-axis is conventionally chosen as the quantisation axis in textbook treatments of angular momentum.
However, in the context of NMR, the $z$-axis bears even more significance as we define it to be the axis along which the static magnetic field is aligned.
Since the matrix elements of an operator $O$ are given by $O_{mn} = \braket{m|O|n}$, we can work out the matrix representations of the angular momentum operators in the Zeeman basis:
\begin{equation}
    \label{eq:pauli}
    I_x = \frac{1}{2}\begin{pmatrix} 0 & 1 \\ 1 & 0 \end{pmatrix}; \quad 
    I_y = \frac{1}{2}\begin{pmatrix} 0 & -\mi \\ \mi & 0 \end{pmatrix}; \quad 
    I_z = \frac{1}{2}\begin{pmatrix} 1 & 0 \\ 0 & -1 \end{pmatrix}
\end{equation}
Their commutators are given by:
\begin{equation}
    \label{eq:angmom_commutators}
    [I_i, I_j] = \sum_{k} \mi \varepsilon_{ijk}J_k,
\end{equation}
where $\varepsilon_{ijk}$ is the Levi-Civita symbol.
We also define the following linear combinations:
\begin{equation}
    % see https://stackoverflow.com/a/2600374/7115316 for equation label alignment
    \label{eq:other_single_spin_ops}
    \begin{aligned}
        I_+ = I_x + \mi I_y &= \begin{pmatrix} 0 & 1 \\ 0 & 0 \end{pmatrix}; &
        I_\alpha = \frac{1}{2}E + I_z &= \begin{pmatrix} 1 & 0 \\ 0 & 0 \end{pmatrix} \\
        I_- = I_x - \mi I_y &= \begin{pmatrix} 0 & 0 \\ 1 & 0 \end{pmatrix}; &
        I_\beta = \frac{1}{2}E - I_z &= \begin{pmatrix} 0 & 0 \\ 0 & 1 \end{pmatrix}
    \end{aligned}
\end{equation}
where $E\/$ is the $2 \times 2$ identity matrix.
The \textit{coherence order} of an operator, denoted $p$, is defined by the Zeeman basis states it connects, i.e.\ the nonzero elements in its matrix form when expressed in this basis: an operator $O = \ket{m_2}\bra{m_1}$ would represent $(m_2 - m_1)$-order coherence, since $\braket{m_2|O|m_1} \neq 0$.
Thus, in the above equations, $I_+ = \ket{\alpha}\bra{\beta}$ represents a coherence order of $+1$; $I_-$ a coherence order of $-1$; $I_x$ and $I_y$ are both a mixture of $\pm 1$-coherence; and the remainder have coherence order $0$.

States (and operators) for composite systems are formally defined as tensor products of single-spin states (and operators)\autocite{Sakurai2021}.
Operators on the same spin commute as per \cref{eq:angmom_commutators}, and operators on different spins fully commute.
The Kronecker product allows these operators to be expressed in matrix form.\autocite{Hore2015}
For example, the operator $2I_xS_z$ can be represented as follows:\footnote{This representation is not unique; it is perfectly possible to reverse the order of the Kronecker product, and as long as this is consistently done, any physically measurable quantities calculated using this alternative will be the same.}
\begin{equation}
    \label{eq:composite_operator}
    2I_xS_z = 2 \cdot \frac{1}{2} \cdot \frac{1}{2} \left[ 
    \begin{pmatrix} 0 & 1 \\ 1 & 0 \end{pmatrix} \otimes
    \begin{pmatrix} 1 & 0 \\ 0 & -1 \end{pmatrix} \right]
    = \frac{1}{2} \begin{pmatrix} 0 & 0 & 1 & 0 \\ 0 & 0 & 0 & -1 \\ 1 & 0 & 0 & 0 \\ 0 & -1 & 0 & 0 \end{pmatrix}
\end{equation}
The Hamiltonians $H\/$ for nuclear spin interactions, which will be encountered frequently in this chapter, are formed from such operators.\autocite{Levitt2008}
In solution-state NMR, these interactions include:
\begin{align}
    \Hcs &= \sum_i \omega_{0,i} I_{iz} & &\text{(chemical shift)} \label{eq:h_cs} \\
    \HJ &= \sum_{i > j} 2\cpi J_{ij} (\symbf{I}_{i}\cdot \symbf{I}_{j}) & &\text{(scalar coupling)} \label{eq:h_j} \\
    \Hpulse &= \sum_i \omega_{i,x}I_{ix} + \sum_i \omega_{i,y}I_{iy} & &\text{(radiofrequency pulses)} \label{eq:h_pulse} \\
    \Hgrad &= \sum_i \gamma_i Gz I_{iz} & &\text{(pulsed field gradients on }z\text{)} \label{eq:h_grad}
\end{align}
Pulsed field gradients (henceforth shortened to \textit{gradients}) can in principle be applied along any axis, not just $z$, but this is dependent on hardware: all the work in this thesis was done on $z$-gradient probes.
In the above expressions:

\begin{itemize}
    \item $\gamma_i$ is the magnetogyric ratio of spin $i$;
    \item $\omega_{0,i}$ refers to the Larmor, or precession, frequency of spin $i$ (usually on the order of \unit{MHz}). The Larmor frequency is defined as
        \begin{equation}
            \label{eq:larmor_frequency}
            \omega_{0,i} = -\gamma_i B_0,
        \end{equation}
        where $B_0$ is the strength of the external (static) magnetic field;
    \item $J_{ij}$ is the scalar coupling constant between spins $i$ and $j$ (expressed in units of \unit{Hz});
    \item $\omega_x$ and $\omega_y$ are amplitudes of radiofrequency (RF) pulses along the $x$- and $y$-axes, which are in general time-dependent, and are related to the so-called $B_1$ by a factor of $\gamma_i$.
    \item $G$ is the amplitude of the gradient, typically in units of \unit{G\per\cm}; and
    \item $z$ is the position of the spin along the $z$-axis, typically in units of \unit{\cm}.
\end{itemize}

Finally, note that in the \textit{weak coupling} regime where
\begin{equation}
    \omega_{0,i} - \omega_{0,j} \gg J_{ij}, \label{eq:weak_coupling}
\end{equation}
the scalar coupling Hamiltonian may be simplified (the \textit{secular approximation}\footnote{This result comes from the use of time-independent nondegenerate perturbation theory: it is based on the assumption that the eigenstates $\{\ket{n}\}$ of the main Hamiltonian $H_0$ are unchanged by the perturbation $V$ (since the first-order correction varies as $\sum_m V_{mn}/(\omega_m - \omega_n) \ll 1$), and only the first-order correction to the energies $E_n^{(1)} = \braket{n | V | n}$ is retained. In this context, $H_0$ and $V$ are respectively $\Hcs$ and $\HJ$. When the condition \cref{eq:weak_coupling} does not hold, the nondegenerate treatment fails; see e.g.\ Sakurai\autocite{Sakurai2021}.}) to
\begin{equation}
    H_{\symup{J}\text{,secular}} = \sum_{i > j} 2\cpi J_{ij} I_{iz}I_{jz} \label{eq:h_j_secular}.
\end{equation}
This condition is always satisfied whenever spins $i$ and $j$ are different nuclides.

Throughout the course of an NMR experiment, RF pulses and gradients are turned on and off, and thus $\Hpulse$ and $\Hgrad$ are time-dependent---although they will always satisfy the `piecewise constant' criterion which allows us to use \cref{eq:time_evolution_piecewise}.
The `free precession' (or simply `free') Hamiltonian, $\Hfree$, refers to the Hamiltonian which is operative whenever no pulses or gradients are being applied: 
\begin{equation}
    \label{eq:h_free}
    \Hfree = \Hcs + \HJ.
\end{equation}
