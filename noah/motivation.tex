\subsection{Motivation}
\label{subsec:noah__genesis_motivation}

The preceding discussion in \cref{sec:noah__introduction} makes clear how supersequences may be constructed from a large variety of modules.
Furthermore, modules which consume (and preserve) the same magnetisation pools may be directly swapped out for one another.
Thus, for example, the \noah{S,C} supersequence can in fact be generalised to any \magn{C} module plus any \magnnot{C} module.
Very broadly speaking, we may define a generic supersequence as having any or all of the following:

\begin{itemize}
    \item an HMBC module, which actually uses \magn{!X} magnetisation but can be placed at the front as discussed in the \noah{B,S,C} example above;
    \item a \magn{N} module;
    \item one or more \magn{C} modules (it is possible to partition the \magn{C} magnetisation pool between two modules, as will be discussed in \cref{subsec:noah__hsqctocsy});
    \item finally, a \magn{!X} module (or a COSY+X combination) which consumes all remaining bulk magnetisation.
\end{itemize}

In the first NOAH paper in 2017\autocite{Kupce2017ACIE}, a total of 285 `viable' supersequences were already listed.
If we further take into account some of the new modules which were developed over the course of my DPhil (\cref{sec:noah__modules}), the generic formula above provides for over 4000 viable supersequences.%
\footnote{This is also ignoring the `parallel' supersequences, which are discussed in \cref{sec:noah__parallel}. The support for parallel supersequences in GENESIS is not complete: integrating these fully would require substantial changes to the user interface, which I have not had time to do.}
(`Non-viable' sequences would be those which have unwanted sensitivity losses, such as the wrongly-ordered \noah{C,S} supersequence.)

In spite of this diversity, \textit{only around 45 pulse programmes} had been made available prior to the GENESIS website (these were distributed either via the supplementary information of NOAH papers, or the Bruker User Library).
The reason for this gap between theory and practice is simple: traditionally, pulse programmes must be written by hand, which is a laborious and fairly error-prone process made worse by the sheer length of NOAH experiments.
Doing this for thousands of supersequence is clearly impractical.
Furthermore, each time a new module is developed, or an old module is improved, updating every relevant supersequence would in itself be a mammoth task.

To bridge this gap, I turned towards the \textit{programmatic} generation of pulse programmes.%
\footnote{This is actually a bit of a lie: GENESIS was initially created \textit{for my own convenience}. Throughout this chapter, I have had to perform many comparisons of different supersequences, and this tool spared me from having to write everything by hand (and---more often than not---subsequently discover mistakes which invalidated the results). Of course, it soon became apparent that it could find much wider use.}
This not only allows users to create their own customised supersequences, but also provides an easy way for pulse sequence updates to be rapidly disseminated to the NMR community, as new pulse programmes can be accessed immediately upon the deployment of updated source code.
Furthermore, the website can serve as a `one-stop' shop where---after downloading pulse programmes---users may download associated NOAH processing scripts and also access instructions on how to run NOAH experiments.
This information did already exist, but was scattered across several different websites and/or journal supplementary information documents, and would have been needlessly confusing to a new user (not to mention the different versions of scripts available in different publications).
