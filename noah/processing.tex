\subsection{Processing improvements}
\label{subsec:noah__genesis_processing}

Finally, I touch on a few small improvements in processing NOAH data.
These are not strictly made possible by the GENESIS website, but having a unified source for pulse programmes and processing scripts does make it substantially easier to propagate these developments to NMR users.


\subsubsection{AU processing scripts}

In the processing of NOAH experiments, the first step is to split up the FIDs of different modules into different files: this is performed by the \texttt{splitx\_au} AU programme.
However, after that is done, \texttt{splitx\_au} \textit{also} calls on auxiliary AU programmes to process each of the resulting datasets.
Each module is associated with a specific auxiliary AU programme: for example, HSQC data are processed with \texttt{noah\_hsqc}, COSY data with \texttt{noah\_cosy}, and so on.

Previously, these auxiliary AU programmes had to be specified manually using the \texttt{USERP1} through \texttt{USERP5} processing parameters, which apply to the first through fifth modules in a supersequence respectively.
Although this was a perfectly serviceable setup, I realised that it was a source of annoyance when running multiple different supersequences, as modules would often be processed incorrectly if these processing parameters were inadvertently copied from an old dataset.
Even when just running a single supersequence, this would nonetheless represent an extra cognitive load, especially for users who are not familiar with NOAH experiments.

Since the pulse programme specifies which modules are run, and the `correct' processing parameters also depend on the modules, it stands to reason that the processing parameters can be encoded in the pulse programme itself.
Specifically, the auxiliary AU programmes are part of the \texttt{NOAHModule} objects (see \cref{lst:module_c_hsqc}), which makes this information accessible to the GENESIS algorithm.
Using this, GENESIS inserts a line near the bottom of every pulse programme, specifying which AU programmes are to be used for each of the modules present.
I furthermore modified the \texttt{splitx\_au} AU programme to parse the pulse programme for this extra information.
The overall effect is that the \texttt{USERP1} through \texttt{USERP5} parameters no longer need to be specified, as long as a GENESIS pulse programme and a recent version of \texttt{splitx\_au} are used.
However, if the parameters \textit{are} specified, they will override the `default' processing instructions found in the pulse programme.
This maintains backwards compatibility with, for example, parameter sets which have been previously saved; it also allows advanced users to customise the processing, if so desired.


\subsubsection{Non-uniform sampling}

Another small improvement relates to the implementation of non-uniform sampling (NUS) in NOAH supersequences.
Because the indirect dimension in NOAH pulse programmes is generated through explicit looping, instead of the \texttt{mc} macro used in most 2D experiments, the standard NUS implementation in TopSpin cannot be used.
Instead, one must explicitly define a list of $t_1$ increments to sample (stored as the \texttt{VCLIST} parameter).
This is true regardless of which looping implementation is used (in \cref{lst:genesis_looping}).

On top of this, the original NUS implementation\autocite{Claridge2019MRC} used a Python script (\texttt{noah\_nus.py}) to modify the NOAH pulse programme such that it only sampled $t_1$ values from this list.
This, however, necessitated storing two copies of the same pulse programme, one with NUS and one without.

There is no way to circumvent the requirement for the \texttt{VCLIST} parameter, but in GENESIS pulse programmes, I opted to implement NUS as an acquisition flag which could be toggled.
This means that there is no need to store two versions of the same pulse programme, and also makes it easier to seamlessly switch between non-uniform and uniform sampling as desired.

Unfortunately, this change is slightly more problematic than the \texttt{splitx\_au} change: this new NUS implementation is not backwards-compatible, so requires a new NUS setup script, which I have called \texttt{noah\_nus2.py}.
This script cannot be used with old pulse programmes, and the old NUS script cannot be used with GENESIS pulse programmes.
Nevertheless, it does represent a real improvement over the previous implementation: the GENESIS website also helps to make the changeover as smooth as possible.%
\footnote{That said, there were still some bugs in the new NUS implementation right up until August 2022, so this change was not as smooth-sailing as I had hoped for.}
