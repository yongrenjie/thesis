\section{Conclusion}
\label{sec:noah__conclusion}

In this chapter, I covered a variety of improvements to NOAH supersequences, including the development of new or improved modules (\cref{sec:noah__modules}), the implementation of basic solvent suppression (\cref{sec:noah__solvsupp}), and the generalised NOAH sequences obtained through `vertical stacking' of different supersequences (\cref{sec:noah__parallel}).

As a technique in fast 2D NMR, NOAH is in my view fairly mature despite its relative recency (the first paper was published in 2017\autocite{Kupce2017ACIE}).
The underlying idea of separating magnetisation pools, and the tools used to do so, have been around for quite a while---most recently, and notably, in the form of ASAP 2D NMR.
Thus, rather than making fundamental conceptual breakthroughs, I would characterise my work here as `polishing the remaining rough edges', whether by removing artefacts or by incorporating slightly more specialised modules.

Naturally, the possibility of using the same concept to accelerate 3D (or higher) experiments is extremely attractive.
Some of the ground work for this has already been laid, in the form of ASAP-type 3D experiments.\autocite{Schanda2009PNMRS}
However, my opinion is that this research topic would extend far beyond NOAH itself: in particular, 3D experiments are often applied to isotopically labelled substances, which lead to a different set of magnetisation pools which are not so easily separable.

At this point, assuming that there are few fundamental breakthroughs to be made in NOAH itself, we must turn to the question of how to increase its uptake.
The accessibility of new techniques is \textit{always} a significant factor, regardless of the merit of the underlying technique.
For this reason, I am particularly proud of the GENESIS website (\cref{sec:noah__genesis}) and the associated improvements in the processing pipeline; the feedback from the NMR community has also been greatly encouraging.
At the same time, there is still some way to go: although one may consider a `standard' NOAH experiment easy enough to set up, after taking into account other desirable features such as automation and NUS, the entire process can become rather non-trivial.

Another important, but under-explored, area is the use of NOAH in the sensitivity-limited regime.
The results in this chapter are derived from samples which are not particularly dilute, and hence belong in the resolution-limited regime.
While this may lead to a maximisation of the \textit{relative} (in that $\rho_t > \rhoteff$), the \textit{absolute} time savings are small because the standalone 2D experiments would be completed fairly quickly anyway.
This is particularly relevant for (e.g.) unstable samples, for which 2D data need to be acquired in as short a time as possible.
Conclusively demonstrating the benefits of NOAH experiments on such samples would go a long way towards convincing more synthetic chemists to adopt the NOAH technique.
