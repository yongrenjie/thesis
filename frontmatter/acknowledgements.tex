\chapter{Acknowledgements}

\epigraph{\singlespacing%
`I cannot make speeches, Emma:'---he soon resumed; and in a tone of such sincere, decided, intelligible tenderness as was tolerably convincing.---`If I loved you less, I might be able to talk about it more.'
}{--- \textsc{Jane Austen}, \textit{Emma}}


First of all, I wish to thank my supervisors, Tim Claridge and Mohammadali Foroozandeh.
Throughout my studies, their guidance has been understated yet unerring; I have become a better scientist and person for it, and hope that this thesis bears witness to the quality of their counsel.
I especially appreciate being afforded the freedom to explore other things which did not directly feed back into my DPhil output.
This includes the short period where I worked in the NMR team, as well as all the time I have spent on programming, which explains---but does not justify---many a week where no work was produced.
Yet, because of this, I have gotten so much more out of these four years than just a DPhil, and can look towards my future career with much optimism.

I am grateful to the Clarendon Fund, as well as the SBM CDT, for funding this work.
They have made it possible for me to stay in Oxford---a place now very dear to me---for far longer than I had expected to.

I am also deeply indebted to my industrial mentors, Peter Howe, Philip Sidebottom, Iain Swan, and Harry Mackenzie.
They have always taken a keen interest in all aspects of my work, and I shall remember with much fondness the very kind invitation to the Jealott's Hill Syngenta site.
Here, I wish to also thank my professional collaborators in the world of magnetic resonance, \EK{} and \JND{}: it has been nothing short of a privilege to work with and learn from them.

Much credit is due to the people whom I have been fortunate to work with in Oxford.
In particular, I wish to thank the NMR staff: Caitlin, James, Maria, Nader, Nick, and Tina (and Tim, again); and various members of the NMR group past and present, including Ali, David, Fay, JB, Jos, Noelle, Sam, and Simon.
By at least occasionally creating in me a desire to go into the office, you have balanced out my naturally sedentary lifestyle, which is no mean feat.
It is my sincere hope that we may continue to stay in touch, now as friends rather than colleagues.

On a more personal level, learning the flute has been my greatest undertaking in Oxford outside of my work: one which has been sometimes stressful, but always joyful.
For this, I must thank my teacher Jean, who has been kind to me well beyond her duty as an instructor, as well as her husband Robin.

I am also very grateful for the friends I have had during my time in Oxford, including (but not limited to) Alvin, Belinda, Jing Long, Liv, Steve, Jieyan, Sheryl, and Wearn Xin; and the rest of my SBM cohort, especially Henry and Kate.
Many of you will already have moved on to other adventures, but I am so glad to have have crossed paths with you at some point.
Distinguished most in both constancy and intimacy, however, is Marie.
I have always enjoyed your company over the last few years:
there are few people who listen to and understand me as you do, and on such a wide range of topics.
It means a great deal to me, certainly more than I let on---thank you so much for everything.

Finally, I come to the people who have immeasurably shaped my life for the better.
My parents have sacrificed much of their freedom to guarantee mine: the work I have done here is entirely due to theirs.
I cannot even begin to repay their unconditional love, but I pray that through my life I may do them proud and thereby vindicate their choices.
Last and dearest is my partner Bernie, who has been the source of every felicity during my time in Oxford.
None of the achievements in this work can equal the profound joy which I derive from her constant presence and unfailing support.
