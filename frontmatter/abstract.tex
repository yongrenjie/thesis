\chapter{Abstract}

Solution-state nuclear magnetic resonance (NMR) spectroscopy is one of the most important analytical techniques in modern organic chemistry.
A series of `core' NMR techniques is widely used for the routine characterisation of organic molecules, ranging from 1D \proton{} and \carbon{} experiments to more complex 2D experiments such as COSY, HSQC, HMBC, and more.
Nonetheless, there remains much room for developments in NMR, especially considering the increased \textit{complexity} and \textit{quantity} of molecules which are being studied.
This thesis describes multiple approaches towards improving the quality, as well as the speed, of NMR data acquisition.

In \cref{chpt:theory}, I first set out the quantum mechanical formalisms required for the analysis of NMR pulse sequences.
The work which follows may be divided into three sections:

\begin{itemize}
    \item \textit{Improved techniques for pure shift NMR} (\cref{chpt:pureshift}).
        In a pure shift experiment, the effects of all homonuclear scalar couplings are suppressed, leading to highly-resolved spectra where each multiplet is collapsed into a singlet.
        A variety of theoretical and experimental approaches are used to search for pure shift methods which have both high sensitivity and low artefact intensity.
        
    \item \textit{On-the-fly optimisation of NMR experiments} (\cref{chpt:poise}).
        Here, I couple NMR acquisition with derivative-free optimisation algorithms which seek to minimise a cost function measured using spectral data.
        The resulting software, called POISE, allows experiment parameters to be tailored specifically for each sample and spectrometer in a fully automated fashion.
        The use of POISE to improve spectral sensitivity and purity is demonstrated on a wide-ranging series of NMR experiments.

    \item \textit{Accelerated 2D NMR data collection} (\cref{chpt:noah}).
        NOAH supersequences allow multiple 2D experiments (`modules') to be acquired in the time of one, through the careful manipulation of different magnetisation pools.
        This chapter contains work on pulse sequence development, including new NOAH-compatible modules, as well as improvements to various pre-existing modules.
        The GENESIS website, which creates Bruker pulse programmes for arbitrary NOAH supersequences, is also described here.
\end{itemize}
