\chapter{Supervisor's foreword}

Solution-state nuclear magnetic resonance (NMR) spectroscopy is one of the most important analytical techniques in modern chemistry and finds widespread use across academia and industry.
Its primary application is in the structure characterisation of novel chemical compounds, for which myriad experiments have been developed that provide increasing levels of structural detail and chemical insight.
Despite its undoubted power and popularity, there remains room for further development in NMR, especially considering the increased complexity of molecules which are being synthesised and increasing demands for higher sample throughput.  

This thesis describes multiple approaches towards improving the quality, speed, and efficiency of NMR data acquisition.
These encompass advances in pulsed NMR techniques themselves as well as new routines for experiment parameter optimisation and pulse sequence scripting.
The methodology and techniques developed provide advanced tools for use across the chemical, pharmaceutical and agrochemical sciences where structure characterisation is a primary requirement.
One of the notable features of the work is also the usability of the new methodology.
In particular, the development of a website for the automated generation of pulse programmes for immediate use on spectrometers, together with the provision of all necessary automation scripts, provides a very user-friendly interface with which to make these techniques readily available to the wider NMR community. 

Some of the outstanding features of the thesis are the depth, quality and clarity of the written work.
Theory is described concisely yet with great precision, and experimental results made clear with beautifully crafted figures.
Every reference is provided with an active link to its associated DOI, adding to the readability of the thesis.
The research carried out by Jonathan Yong has generated 6 journal publications, in addition to which he has contributed to a major review article on parallel NMR methodology and a book chapter describing fast NMR techniques.
These endeavours and the exceptional work presented herein are certainly worthy of the Springer Thesis Prize. 

\vspace{0.3cm}

\hfuzz=30pt
\begin{tabularx}{\textwidth}{@{}lXr@{}}
    29th August 2023 & & Prof.\ Tim Claridge
\end{tabularx}
