\usepackage{fontspec}
\usepackage{amsmath}
\usepackage{mathtools}
\usepackage[warnings-off={mathtools-colon,mathtools-overbracket}]{unicode-math}
\defaultfontfeatures{Ligatures=TeX}

% Main fonts used in the document
\setmainfont[
  Path={/Users/yongrenjie/Library/Fonts/},
  Extension=.otf,
  UprightFont={*-regular},
  BoldFont={*-semibold},
  ItalicFont={*-italic},
  BoldItalicFont={*-semibolditalic},
]{minion3}
\setmonofont[
  Path={/Users/yongrenjie/Library/Fonts/},
  Extension=.ttf,
  UprightFont={*-Regular},
  BoldFont={*-SemiBold},
  ItalicFont={*-Italic},
  BoldItalicFont={*-SemiBoldItalic},
  Scale=MatchLowercase
]{RobotoMono}
\setmathfont[
  Path={./fonts/},
  Scale=MatchLowercase
]{LibertinusMath-Regular.otf}
\mathitalicsmode=1

% Other optical sizes for Minion
\newfontfamily{\fontcaption}{minion3caption}[
  Path={/Users/yongrenjie/Library/Fonts/},
  Extension=.otf,
  UprightFont={*-regular},
  BoldFont={*-semibold},
  ItalicFont={*-italic},
  BoldItalicFont={*-semibolditalic},
]
\newfontfamily{\fontsubhead}{minion3subhead}[
  Path={/Users/yongrenjie/Library/Fonts/},
  Extension=.otf,
  UprightFont={*-regular},
  BoldFont={*-bold},
  ItalicFont={*-italic},
  BoldItalicFont={*-bolditalic},
]

\usepackage{microtype}
